\documentclass[journal,12pt,twocolumn]{IEEEtran}
%
\usepackage{setspace}
\usepackage{gensymb}
%\doublespacing
\singlespacing

%\usepackage{graphicx}
%\usepackage{amssymb}
%\usepackage{relsize}
\usepackage[cmex10]{amsmath}
%\usepackage{amsthm}
%\interdisplaylinepenalty=2500
%\savesymbol{iint}
%\usepackage{txfonts}
%\restoresymbol{TXF}{iint}
%\usepackage{wasysym}
\usepackage{amsthm}
%\usepackage{iithtlc}
\usepackage{mathrsfs}
\usepackage{txfonts}
\usepackage{stfloats}
\usepackage{bm}
\usepackage{cite}
\usepackage{cases}
\usepackage{subfig}
%\usepackage{xtab}
\usepackage{longtable}
\usepackage{multirow}
%\usepackage{algorithm}
%\usepackage{algpseudocode}
\usepackage[utf8]{inputenc}
\usepackage{tikz}
\usetikzlibrary{positioning}
\usepackage{enumitem}
\usepackage{mathtools}
\usepackage{steinmetz}
\usepackage{tikz}
\usepackage{circuitikz}
\usepackage{verbatim}
\usepackage{tfrupee}
\usepackage[breaklinks=true]{hyperref}
%\usepackage{stmaryrd}
\usepackage{tkz-euclide} % loads  TikZ and tkz-base
%\usetkzobj{all}
\usetikzlibrary{calc,math}
\usepackage{listings}
    \usepackage{color}                                            %%
    \usepackage{array}                                            %%
    \usepackage{longtable}                                        %%
    \usepackage{calc}                                             %%
    \usepackage{multirow}                                         %%
    \usepackage{hhline}                                           %%
    \usepackage{ifthen}                                           %%
  %optionally (for landscape tables embedded in another document): %%
    \usepackage{lscape}     
\usepackage{multicol}
\usepackage{chngcntr}
%\usepackage{enumerate}

%\usepackage{wasysym}
%\newcounter{MYtempeqncnt}
\DeclareMathOperator*{\Res}{Res}
%\renewcommand{\baselinestretch}{2}
\renewcommand\thesection{\arabic{section}}
\renewcommand\thesubsection{\thesection.\arabic{subsection}}
\renewcommand\thesubsubsection{\thesubsection.\arabic{subsubsection}}

\renewcommand\thesectiondis{\arabic{section}}
\renewcommand\thesubsectiondis{\thesectiondis.\arabic{subsection}}
\renewcommand\thesubsubsectiondis{\thesubsectiondis.\arabic{subsubsection}}

% correct bad hyphenation here
\hyphenation{op-tical net-works semi-conduc-tor}
\def\inputGnumericTable{}                                 %%

\lstset{
%language=C,
frame=single, 
breaklines=true,
columns=fullflexible
}
%\lstset{
%language=tex,
%frame=single, 
%breaklines=true
%}

\begin{document}
%


\newtheorem{theorem}{Theorem}[section]
\newtheorem{problem}{Problem}
\newtheorem{proposition}{Proposition}[section]
\newtheorem{lemma}{Lemma}[section]
\newtheorem{corollary}[theorem]{Corollary}
\newtheorem{example}{Example}[section]
\newtheorem{definition}[problem]{Definition}
%\newtheorem{thm}{Theorem}[section] 
%\newtheorem{defn}[thm]{Definition}
%\newtheorem{algorithm}{Algorithm}[section]
%\newtheorem{cor}{Corollary}
\newcommand{\BEQA}{\begin{eqnarray}}
\newcommand{\EEQA}{\end{eqnarray}}
\newcommand{\define}{\stackrel{\triangle}{=}}
\bibliographystyle{IEEEtran}
%\bibliographystyle{ieeetr}
\providecommand{\mbf}{\mathbf}
\providecommand{\pr}[1]{\ensuremath{\Pr\left(#1\right)}}
\providecommand{\qfunc}[1]{\ensuremath{Q\left(#1\right)}}
\providecommand{\sbrak}[1]{\ensuremath{{}\left[#1\right]}}
\providecommand{\lsbrak}[1]{\ensuremath{{}\left[#1\right.}}
\providecommand{\rsbrak}[1]{\ensuremath{{}\left.#1\right]}}
\providecommand{\brak}[1]{\ensuremath{\left(#1\right)}}
\providecommand{\lbrak}[1]{\ensuremath{\left(#1\right.}}
\providecommand{\rbrak}[1]{\ensuremath{\left.#1\right)}}
\providecommand{\cbrak}[1]{\ensuremath{\left\{#1\right\}}}
\providecommand{\lcbrak}[1]{\ensuremath{\left\{#1\right.}}
\providecommand{\rcbrak}[1]{\ensuremath{\left.#1\right\}}}
\theoremstyle{remark}
\newtheorem{rem}{Remark}
\newcommand{\sgn}{\mathop{\mathrm{sgn}}}
\providecommand{\abs}[1]{\ensuremath{\left\vert#1\right\vert}}
\providecommand{\res}[1]{\Res\displaylimits_{#1}} 
\providecommand{\norm}[1]{\ensuremath{\left\lVert#1\right\rVert}}
%\providecommand{\norm}[1]{\lVert#1\rVert}
\providecommand{\mtx}[1]{\mathbf{#1}}
\providecommand{\mean}[1]{\ensuremath{E\left[ #1 \right]}}
\providecommand{\fourier}{\overset{\mathcal{F}}{ \rightleftharpoons}}
%\providecommand{\hilbert}{\overset{\mathcal{H}}{ \rightleftharpoons}}
\providecommand{\system}{\overset{\mathcal{H}}{ \longleftrightarrow}}
	%\newcommand{\solution}[2]{\textbf{Solution:}{#1}}
\newcommand{\solution}{\noindent \textbf{Solution: }}
\newcommand{\cosec}{\,\text{cosec}\,}
\providecommand{\dec}[2]{\ensuremath{\overset{#1}{\underset{#2}{\gtrless}}}}
\newcommand{\myvec}[1]{\ensuremath{\begin{pmatrix}#1\end{pmatrix}}}
\newcommand{\mydet}[1]{\ensuremath{\begin{vmatrix}#1\end{vmatrix}}}
%\numberwithin{equation}{section}
\numberwithin{equation}{subsection}
%\numberwithin{problem}{section}
%\numberwithin{definition}{section}
\makeatletter
\@addtoreset{figure}{problem}
\makeatother
\let\StandardTheFigure\thefigure
\let\vec\mathbf
%\renewcommand{\thefigure}{\theproblem.\arabic{figure}}
\renewcommand{\thefigure}{\theproblem}
%\setlist[enumerate,1]{before=\renewcommand\theequation{\theenumi.\arabic{equation}}
%\counterwithin{equation}{enumi}
%\renewcommand{\theequation}{\arabic{subsection}.\arabic{equation}}
\def\putbox#1#2#3{\makebox[0in][l]{\makebox[#1][l]{}\raisebox{\baselineskip}[0in][0in]{\raisebox{#2}[0in][0in]{#3}}}}
     \def\rightbox#1{\makebox[0in][r]{#1}}
     \def\centbox#1{\makebox[0in]{#1}}
     \def\topbox#1{\raisebox{-\baselineskip}[0in][0in]{#1}}
     \def\midbox#1{\raisebox{-0.5\baselineskip}[0in][0in]{#1}}
\vspace{3cm}
\title{Assignment 10}
\author{Sairam V C Rebbapragada}
\maketitle
\newpage
%\tableofcontents
\bigskip
\renewcommand{\thefigure}{\theenumi}
\renewcommand{\thetable}{\theenumi}
\begin{abstract}
This document uses the concepts of representation of transformations by matrices in solving a problem.
\end{abstract}
Download Python code from 
%
\begin{lstlisting}
https://github.com/Sairam13001/AI5106/blob/main/Assignment_9/assignment_10.py
\end{lstlisting}
%
Download latex-tikz codes from 
%
\begin{lstlisting}
https://github.com/Sairam13001/AI5106/blob/main/Assignment_9/assignment_10.tex
\end{lstlisting}
%

\section{Problem}
Let $\mathbf{T}$ be the linear operator on $\mathbb{R}^2$ defined by
\begin{align}
    T\brak{x_1, x_2} = \brak{-x_2, x_1} \label{given_lin_op}
\end{align}
Prove that if $\vec{\beta}$ is any ordered basis for $\mathbb{R}^2$ and $[\vec{T}]_\beta = \vec{A}$ then $\vec{A}_{12}\vec{A}_{21} \neq 0$.  
\section{Explanation}
If the matrix $\myvec{a & b \\ c & d}$ is invertible, then $ad-bc \neq 0$. It implies that both a and b cannot become 0 at the same time.
\section{Solution}
\begin{align}
    \text{Let } \vec{x} &= \myvec{x_1 \\ x_2} \\
    T(\vec{x}) &= A\vec{x}\\
     &= \myvec{0 & -1 \\ 1 & 0}\myvec{x_1 \\ x_2}\\
     \implies \vec{A} &= \myvec{0 & -1 \\ 1 & 0}
\end{align}
Let $\{\alpha_1,\alpha_2\}$ be any basis and $\alpha_1=\brak{a,b}$ and $\alpha_2=\brak{c,d}$. Then 
\begin{align}
    T(\beta) &= \vec{A}\beta\\ 
    T(\beta) &= \myvec{-b & -d \\ a & c} 
\end{align}
Now, for finding the matrix of $\mathbf{T}$ in the ordered basis $\beta$, We use the concept of row reducing the augmented matrix.
\begin{align}
    \myvec{a & c & -b & -d \\ b & d & a & c}
\end{align}
\begin{enumerate}
    \item Assuming $a \neq 0$, 
    \begin{align}
        &\myvec{a & c & -b & -d \\ b & d & a & c}  \nonumber \\
        &\xleftrightarrow{R_1 = R_1/a} \myvec{1 & \frac{c}{a} & -\frac{b}{a} & -\frac{d}{a} \\ b & d & a & c} \nonumber \\
&\xleftrightarrow{R_2 = R_2 - R_1\times b} \myvec{1 & \frac{c}{a} & -\frac{b}{a} & -\frac{d}{a} \\ 0 & \frac{ad-bc}{a} & \frac{a^2+b^2}{a} & \frac{ac+bd}{a}} \nonumber \\
&\xleftrightarrow{R_2 = R_2\frac{a}{ad-bc}} \myvec{1 & \frac{c}{a} & -\frac{b}{a} & -\frac{d}{a} \\ 0 & 1 & \frac{a^2+b^2}{ad-bc} & \frac{ac+bd}{ad-bc}} \nonumber \\
&\xleftrightarrow{R_1 = R_1 - R_2\times\frac{c}{a}} \myvec{1 & 0 & -\frac{ac+bd}{ad-bc} & -\frac{c^2+d^2}{ad-bc} \\ 0 & 1 & \frac{a^2+b^2}{ad-bc} & \frac{ac+bd}{ad-bc}} \nonumber
    \end{align}
\item Assuming $b \neq 0$, We arrive at the same result as follows
\begin{align}
    &\myvec{a & c & -b & -d \\ b & d & a & c} \nonumber \\ 
    &\xleftrightarrow{R_1\leftrightarrow R_2} \myvec{ b & d & a & c \\ a & c & -b & -d } \nonumber \\
&\xleftrightarrow{R_1 = R_1/b}   \myvec{ 1 & \frac{d}{b} & \frac{a}{b} & \frac{c}{b} \\ a & c & -b & -d } \nonumber \\
&\xleftrightarrow{R_2 = R_2 - R_1\times a} \myvec{ 1 & \frac{d}{b} & \frac{a}{b} & \frac{c}{b} \\ 0 & \frac{bc-ad}{b} & -\frac{a^2+b^2}{b} & -\frac{ac+bd}{b}} \nonumber \\
&\xleftrightarrow{R_2 = R_2\frac{b}{bc-ad}} \myvec{ 1 & \frac{d}{b} & \frac{a}{b} & \frac{c}{b} \\ 0 & 1 & \frac{a^2+b^2}{ad-bc} & \frac{ac+bd}{ad-bc}} \nonumber \\
&\xleftrightarrow{R_1 = R_1 - R_2\times\frac{d}{b}} \myvec{1 & 0 & -\frac{ac+bd}{ad-bc} & -\frac{c^2+d^2}{ad-bc} \\ 0 & 1 & \frac{a^2+b^2}{ad-bc} & \frac{ac+bd}{ad-bc}} \nonumber
\end{align}
\end{enumerate}
Thus, from above two cases:
\begin{align}
    [\vec{T}]_\beta = \vec{A} = \myvec{-\frac{ac+bd}{ad-bc} & -\frac{c^2+d^2}{ad-bc} \\ \frac{a^2+b^2}{ad-bc} & \frac{ac+bd}{ad-bc}}.
\end{align}
Now $ad-bc \neq 0$ implies that atleast one of a or b is non-zero and atleast one of c or d is non-zero, which follows that:
\begin{align}
    a^2 + b^2 > 0 \text{ and } c^2 + d^2 > 0 \\
    \text{Thus, } \brak{a^2+b^2}\brak{c^2+d^2} \neq 0.
\end{align}
Hence, it is proved that $\vec{A}_{12}\vec{A}_{21} \neq 0$.
\end{document}
