\documentclass[journal,12pt,twocolumn]{IEEEtran}
%
\usepackage{setspace}
\usepackage{gensymb}
%\doublespacing
\singlespacing

%\usepackage{graphicx}
%\usepackage{amssymb}
%\usepackage{relsize}
\usepackage[cmex10]{amsmath}
%\usepackage{amsthm}
%\interdisplaylinepenalty=2500
%\savesymbol{iint}
%\usepackage{txfonts}
%\restoresymbol{TXF}{iint}
%\usepackage{wasysym}
\usepackage{amsthm}
%\usepackage{iithtlc}
\usepackage{mathrsfs}
\usepackage{txfonts}
\usepackage{stfloats}
\usepackage{bm}
\usepackage{cite}
\usepackage{cases}
\usepackage{subfig}
%\usepackage{xtab}
\usepackage{longtable}
\usepackage{multirow}
%\usepackage{algorithm}
%\usepackage{algpseudocode}
\usepackage[utf8]{inputenc}
\usepackage{tikz}
\usetikzlibrary{positioning}
\usepackage{enumitem}
\usepackage{mathtools}
\usepackage{steinmetz}
\usepackage{tikz}
\usepackage{circuitikz}
\usepackage{verbatim}
\usepackage{tfrupee}
\usepackage[breaklinks=true]{hyperref}
%\usepackage{stmaryrd}
\usepackage{tkz-euclide} % loads  TikZ and tkz-base
%\usetkzobj{all}
\usetikzlibrary{calc,math}
\usepackage{listings}
    \usepackage{color}                                            %%
    \usepackage{array}                                            %%
    \usepackage{longtable}                                        %%
    \usepackage{calc}                                             %%
    \usepackage{multirow}                                         %%
    \usepackage{hhline}                                           %%
    \usepackage{ifthen}                                           %%
  %optionally (for landscape tables embedded in another document): %%
    \usepackage{lscape}     
\usepackage{multicol}
\usepackage{chngcntr}
%\usepackage{enumerate}

%\usepackage{wasysym}
%\newcounter{MYtempeqncnt}
\DeclareMathOperator*{\Res}{Res}
%\renewcommand{\baselinestretch}{2}
\renewcommand\thesection{\arabic{section}}
\renewcommand\thesubsection{\thesection.\arabic{subsection}}
\renewcommand\thesubsubsection{\thesubsection.\arabic{subsubsection}}

\renewcommand\thesectiondis{\arabic{section}}
\renewcommand\thesubsectiondis{\thesectiondis.\arabic{subsection}}
\renewcommand\thesubsubsectiondis{\thesubsectiondis.\arabic{subsubsection}}

% correct bad hyphenation here
\hyphenation{op-tical net-works semi-conduc-tor}
\def\inputGnumericTable{}                                 %%

\lstset{
%language=C,
frame=single, 
breaklines=true,
columns=fullflexible
}
%\lstset{
%language=tex,
%frame=single, 
%breaklines=true
%}

\begin{document}
%


\newtheorem{theorem}{Theorem}[section]
\newtheorem{problem}{Problem}
\newtheorem{proposition}{Proposition}[section]
\newtheorem{lemma}{Lemma}[section]
\newtheorem{corollary}[theorem]{Corollary}
\newtheorem{example}{Example}[section]
\newtheorem{definition}[problem]{Definition}
%\newtheorem{thm}{Theorem}[section] 
%\newtheorem{defn}[thm]{Definition}
%\newtheorem{algorithm}{Algorithm}[section]
%\newtheorem{cor}{Corollary}
\newcommand{\BEQA}{\begin{eqnarray}}
\newcommand{\EEQA}{\end{eqnarray}}
\newcommand{\define}{\stackrel{\triangle}{=}}
\bibliographystyle{IEEEtran}
%\bibliographystyle{ieeetr}
\providecommand{\mbf}{\mathbf}
\providecommand{\pr}[1]{\ensuremath{\Pr\left(#1\right)}}
\providecommand{\qfunc}[1]{\ensuremath{Q\left(#1\right)}}
\providecommand{\sbrak}[1]{\ensuremath{{}\left[#1\right]}}
\providecommand{\lsbrak}[1]{\ensuremath{{}\left[#1\right.}}
\providecommand{\rsbrak}[1]{\ensuremath{{}\left.#1\right]}}
\providecommand{\brak}[1]{\ensuremath{\left(#1\right)}}
\providecommand{\lbrak}[1]{\ensuremath{\left(#1\right.}}
\providecommand{\rbrak}[1]{\ensuremath{\left.#1\right)}}
\providecommand{\cbrak}[1]{\ensuremath{\left\{#1\right\}}}
\providecommand{\lcbrak}[1]{\ensuremath{\left\{#1\right.}}
\providecommand{\rcbrak}[1]{\ensuremath{\left.#1\right\}}}
\theoremstyle{remark}
\newtheorem{rem}{Remark}
\newcommand{\sgn}{\mathop{\mathrm{sgn}}}
\providecommand{\abs}[1]{\ensuremath{\left\vert#1\right\vert}}
\providecommand{\res}[1]{\Res\displaylimits_{#1}} 
\providecommand{\norm}[1]{\ensuremath{\left\lVert#1\right\rVert}}
%\providecommand{\norm}[1]{\lVert#1\rVert}
\providecommand{\mtx}[1]{\mathbf{#1}}
\providecommand{\mean}[1]{\ensuremath{E\left[ #1 \right]}}
\providecommand{\fourier}{\overset{\mathcal{F}}{ \rightleftharpoons}}
%\providecommand{\hilbert}{\overset{\mathcal{H}}{ \rightleftharpoons}}
\providecommand{\system}{\overset{\mathcal{H}}{ \longleftrightarrow}}
	%\newcommand{\solution}[2]{\textbf{Solution:}{#1}}
\newcommand{\solution}{\noindent \textbf{Solution: }}
\newcommand{\cosec}{\,\text{cosec}\,}
\providecommand{\dec}[2]{\ensuremath{\overset{#1}{\underset{#2}{\gtrless}}}}
\newcommand{\myvec}[1]{\ensuremath{\begin{pmatrix}#1\end{pmatrix}}}
\newcommand{\mydet}[1]{\ensuremath{\begin{vmatrix}#1\end{vmatrix}}}
%\numberwithin{equation}{section}
\numberwithin{equation}{subsection}
%\numberwithin{problem}{section}
%\numberwithin{definition}{section}
\makeatletter
\@addtoreset{figure}{problem}
\makeatother
\let\StandardTheFigure\thefigure
\let\vec\mathbf
%\renewcommand{\thefigure}{\theproblem.\arabic{figure}}
\renewcommand{\thefigure}{\theproblem}
%\setlist[enumerate,1]{before=\renewcommand\theequation{\theenumi.\arabic{equation}}
%\counterwithin{equation}{enumi}
%\renewcommand{\theequation}{\arabic{subsection}.\arabic{equation}}
\def\putbox#1#2#3{\makebox[0in][l]{\makebox[#1][l]{}\raisebox{\baselineskip}[0in][0in]{\raisebox{#2}[0in][0in]{#3}}}}
     \def\rightbox#1{\makebox[0in][r]{#1}}
     \def\centbox#1{\makebox[0in]{#1}}
     \def\topbox#1{\raisebox{-\baselineskip}[0in][0in]{#1}}
     \def\midbox#1{\raisebox{-0.5\baselineskip}[0in][0in]{#1}}
\vspace{3cm}
\title{Assignment 7}
\author{Sairam V C Rebbapragada}
\maketitle
\newpage
%\tableofcontents
\bigskip
\renewcommand{\thefigure}{\theenumi}
\renewcommand{\thetable}{\theenumi}
\begin{abstract}
This document uses  Gram-Schmidt process to perform QR decomposition of a matrix.
\end{abstract}
Download Python code from 
%
\begin{lstlisting}
https://github.com/Sairam13001/AI5106/blob/main/Assignment_7/assignment_7.py
\end{lstlisting}
%
Download latex-tikz codes from 
%
\begin{lstlisting}
https://github.com/Sairam13001/AI5106/blob/main/Assignment_7/assignment_7.tex
\end{lstlisting}
%

\section{Problem}
\begin{align}
14x^2 - 4xy + 11y^2 - 44x - 58y + 71 =0  \label{given_curve_eq}
\end{align}
Express this conic section in the from
\begin{align}
\vec{x}^T\vec{V}\vec{x}+2\vec{u}^T\vec{x}+f=0 \label{conic_quad_eqn}
\end{align} 
and perform QR decomposition for the matrix V.

\section{Explanation}

The general equation of second degree is given by
\begin{align}
ax^2+2bxy+cy^2+2dx+2ey+f=0 \label{gen_quad_eqn}
\end{align}
and can be expressed as
\begin{align}
\vec{x}^T\vec{V}\vec{x}+2\vec{u}^T\vec{x}+f=0 \label{conic_quad_eqn}
\end{align}
where
\begin{align}
\vec{V} &= \vec{V}^T = \myvec{a & b \\ b & c}
\end{align}
The QR decomposition (also called the QR factorization) of a matrix is a decomposition of the matrix into an orthogonal matrix and a triangular matrix. A QR decomposition of a real square matrix $\vec{V}$ is a decomposition of $\vec{V}$ as $\vec{V}$ = $\vec{QR}$,
where $\vec{Q}$ is an orthogonal matrix $\brak{\vec{Q}^T\vec{Q} = \vec{I}}$ and $\vec{R}$ is an upper triangular matrix.
\section{Solution}

Comparing \eqref{given_curve_eq} with \eqref{gen_quad_eqn}, we get
\begin{align}
\vec{V} &= \myvec{14  & -2 \\ -2 & 11}
\end{align}
\begin{align}
\text{If, } \vec{V} = \myvec{a_1  & b_1 \\ a_2 & b_2} 
\text{ Consider, } \vec{V} = \myvec{\vec{a} & \vec{b}} \nonumber \\
\text{Where, } \vec{a} = \myvec{a1 \\ a2} \text{and } \vec{b} = \myvec{b1 \\ b2} 
\end{align}
\begin{align}
\text{Then, } \vec{u_1} &= \vec{a} \text{ , } \vec{e_1} = \frac{\vec{u_1}}{\norm{\vec{u_1}}}  \nonumber \\
\vec{u_2} &= \vec{b} - \brak{\vec{b}\cdot\vec{e_1}}\vec{e_1} \text{ , } \vec{e_2} = \frac{\vec{u_2}}{\norm{\vec{u_2}}} 
\end{align}
\begin{align}
\text{and, } \vec{V} = \myvec{\vec{e_1} & \vec{e_2}}\myvec{\vec{a}\cdot\vec{e_1} & \vec{b}\cdot\vec{e_1} \\ 0 & \vec{b}\cdot\vec{e_2}} = \vec{Q}\vec{R}
\end{align}

Performing QR decomposition on $\vec{V}$ we get,
\begin{align}
\vec{e_1} &= \frac{1}{10\sqrt{2}}\myvec{14 \\ -2}  \\
\vec{e_2} &= \frac{1}{\sqrt{50}}\myvec{1 \\ 7}  \\
\vec{Q} &= \myvec{\vec{e_1} & \vec{e_2}} \\
\vec{R} &= \myvec{10\sqrt{2} & -5\sqrt{2} \\ 0 & \frac{75}{\sqrt{50}}} 
\end{align}

It is easy to verify that $\vec{Q}\vec{R} = \vec{V}$ and $\vec{Q}^T\vec{Q} = \vec{I}$. Thus, $\vec{V}$  is decomposed into an orthogonal matrix and an upper triangular matrix.
\end{document}
