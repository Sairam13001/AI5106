\documentclass[journal,12pt,twocolumn]{IEEEtran}
%
\usepackage{setspace}
\usepackage{gensymb}
%\doublespacing
\singlespacing

%\usepackage{graphicx}
%\usepackage{amssymb}
%\usepackage{relsize}
\usepackage[cmex10]{amsmath}
%\usepackage{amsthm}
%\interdisplaylinepenalty=2500
%\savesymbol{iint}
%\usepackage{txfonts}
%\restoresymbol{TXF}{iint}
%\usepackage{wasysym}
\usepackage{amsthm}
%\usepackage{iithtlc}
\usepackage{mathrsfs}
\usepackage{txfonts}
\usepackage{stfloats}
\usepackage{bm}
\usepackage{cite}
\usepackage{cases}
\usepackage{subfig}
%\usepackage{xtab}
\usepackage{longtable}
\usepackage{multirow}
%\usepackage{algorithm}
%\usepackage{algpseudocode}
\usepackage[utf8]{inputenc}
\usepackage{tikz}
\usetikzlibrary{positioning}
\usepackage{enumitem}
\usepackage{mathtools}
\usepackage{steinmetz}
\usepackage{tikz}
\usepackage{circuitikz}
\usepackage{verbatim}
\usepackage{tfrupee}
\usepackage[breaklinks=true]{hyperref}
%\usepackage{stmaryrd}
\usepackage{tkz-euclide} % loads  TikZ and tkz-base
%\usetkzobj{all}
\usetikzlibrary{calc,math}
\usepackage{listings}
    \usepackage{color}                                            %%
    \usepackage{array}                                            %%
    \usepackage{longtable}                                        %%
    \usepackage{calc}                                             %%
    \usepackage{multirow}                                         %%
    \usepackage{hhline}                                           %%
    \usepackage{ifthen}                                           %%
  %optionally (for landscape tables embedded in another document): %%
    \usepackage{lscape}     
\usepackage{multicol}
\usepackage{chngcntr}
%\usepackage{enumerate}

%\usepackage{wasysym}
%\newcounter{MYtempeqncnt}
\DeclareMathOperator*{\Res}{Res}
%\renewcommand{\baselinestretch}{2}
\renewcommand\thesection{\arabic{section}}
\renewcommand\thesubsection{\thesection.\arabic{subsection}}
\renewcommand\thesubsubsection{\thesubsection.\arabic{subsubsection}}

\renewcommand\thesectiondis{\arabic{section}}
\renewcommand\thesubsectiondis{\thesectiondis.\arabic{subsection}}
\renewcommand\thesubsubsectiondis{\thesubsectiondis.\arabic{subsubsection}}

% correct bad hyphenation here
\hyphenation{op-tical net-works semi-conduc-tor}
\def\inputGnumericTable{}                                 %%

\lstset{
%language=C,
frame=single, 
breaklines=true,
columns=fullflexible
}
%\lstset{
%language=tex,
%frame=single, 
%breaklines=true
%}

\begin{document}
%


\newtheorem{theorem}{Theorem}[section]
\newtheorem{problem}{Problem}
\newtheorem{proposition}{Proposition}[section]
\newtheorem{lemma}{Lemma}[section]
\newtheorem{corollary}[theorem]{Corollary}
\newtheorem{example}{Example}[section]
\newtheorem{definition}[problem]{Definition}
%\newtheorem{thm}{Theorem}[section] 
%\newtheorem{defn}[thm]{Definition}
%\newtheorem{algorithm}{Algorithm}[section]
%\newtheorem{cor}{Corollary}
\newcommand{\BEQA}{\begin{eqnarray}}
\newcommand{\EEQA}{\end{eqnarray}}
\newcommand{\define}{\stackrel{\triangle}{=}}
\bibliographystyle{IEEEtran}
%\bibliographystyle{ieeetr}
\providecommand{\mbf}{\mathbf}
\providecommand{\pr}[1]{\ensuremath{\Pr\left(#1\right)}}
\providecommand{\qfunc}[1]{\ensuremath{Q\left(#1\right)}}
\providecommand{\sbrak}[1]{\ensuremath{{}\left[#1\right]}}
\providecommand{\lsbrak}[1]{\ensuremath{{}\left[#1\right.}}
\providecommand{\rsbrak}[1]{\ensuremath{{}\left.#1\right]}}
\providecommand{\brak}[1]{\ensuremath{\left(#1\right)}}
\providecommand{\lbrak}[1]{\ensuremath{\left(#1\right.}}
\providecommand{\rbrak}[1]{\ensuremath{\left.#1\right)}}
\providecommand{\cbrak}[1]{\ensuremath{\left\{#1\right\}}}
\providecommand{\lcbrak}[1]{\ensuremath{\left\{#1\right.}}
\providecommand{\rcbrak}[1]{\ensuremath{\left.#1\right\}}}
\theoremstyle{remark}
\newtheorem{rem}{Remark}
\newcommand{\sgn}{\mathop{\mathrm{sgn}}}
\providecommand{\abs}[1]{\ensuremath{\left\vert#1\right\vert}}
\providecommand{\res}[1]{\Res\displaylimits_{#1}} 
\providecommand{\norm}[1]{\ensuremath{\left\lVert#1\right\rVert}}
%\providecommand{\norm}[1]{\lVert#1\rVert}
\providecommand{\mtx}[1]{\mathbf{#1}}
\providecommand{\mean}[1]{\ensuremath{E\left[ #1 \right]}}
\providecommand{\fourier}{\overset{\mathcal{F}}{ \rightleftharpoons}}
%\providecommand{\hilbert}{\overset{\mathcal{H}}{ \rightleftharpoons}}
\providecommand{\system}{\overset{\mathcal{H}}{ \longleftrightarrow}}
	%\newcommand{\solution}[2]{\textbf{Solution:}{#1}}
\newcommand{\solution}{\noindent \textbf{Solution: }}
\newcommand{\cosec}{\,\text{cosec}\,}
\providecommand{\dec}[2]{\ensuremath{\overset{#1}{\underset{#2}{\gtrless}}}}
\newcommand{\myvec}[1]{\ensuremath{\begin{pmatrix}#1\end{pmatrix}}}
\newcommand{\mydet}[1]{\ensuremath{\begin{vmatrix}#1\end{vmatrix}}}
%\numberwithin{equation}{section}
\numberwithin{equation}{subsection}
%\numberwithin{problem}{section}
%\numberwithin{definition}{section}
\makeatletter
\@addtoreset{figure}{problem}
\makeatother
\let\StandardTheFigure\thefigure
\let\vec\mathbf
%\renewcommand{\thefigure}{\theproblem.\arabic{figure}}
\renewcommand{\thefigure}{\theproblem}
%\setlist[enumerate,1]{before=\renewcommand\theequation{\theenumi.\arabic{equation}}
%\counterwithin{equation}{enumi}
%\renewcommand{\theequation}{\arabic{subsection}.\arabic{equation}}
\def\putbox#1#2#3{\makebox[0in][l]{\makebox[#1][l]{}\raisebox{\baselineskip}[0in][0in]{\raisebox{#2}[0in][0in]{#3}}}}
     \def\rightbox#1{\makebox[0in][r]{#1}}
     \def\centbox#1{\makebox[0in]{#1}}
     \def\topbox#1{\raisebox{-\baselineskip}[0in][0in]{#1}}
     \def\midbox#1{\raisebox{-0.5\baselineskip}[0in][0in]{#1}}
\vspace{3cm}
\title{Assignment 12}
\author{Sairam V C Rebbapragada}
\maketitle
\newpage
%\tableofcontents
\bigskip
\renewcommand{\thefigure}{\theenumi}
\renewcommand{\thetable}{\theenumi}
\begin{abstract}
This document uses the concepts of matrix trace, eigen values, minimal polynomial etc in solving a problem. 
\end{abstract}

Download latex-tikz codes from 
%
\begin{lstlisting}
https://github.com/Sairam13001/AI5106/blob/main/Assignment_12/assignment_12.tex
\end{lstlisting}

\section{Problem}
Let $M_n(\mathbb{R})$ be the ring of $n \times n$ matrices over $\mathbb{R}$. Which of the following are true for every $n\geq 2$?
\begin{enumerate}
    \item there exists matrices A, B $\in$ $M_n(\mathbb{R})$ such that $AB - BA = I_n$, where $I_n$ denotes the identity $n \times n$ matrix.
    \item if A, B $\in$ $M_n(\mathbb{R})$ and $AB = BA$, then A is diagonalizable over $\mathbb{R}$ if and only if B is diagonalizable over $\mathbb{R}$.
    \item if A, B $\in$ $M_n(\mathbb{R})$, then $AB$ and $BA$ have same minimal polynomial.
    \item if A, B $\in$ $M_n(\mathbb{R})$, then $AB$ and $BA$ have same eigenvalues in $\mathbb{R}$.
\end{enumerate}
\section{Explanation}
If A, B are two matrices of same size and k is any scalar, then we know that 
\begin{align}
    trace(A+B) &= trace(A) + trace(B)\\
    trace(kA) &= k \times trace(A) \\
    trace(AB) &= trace(BA)
\end{align}
\section{Solution}
\begin{enumerate}
    \item Consider option 1 is true. It means that there exists matrices A, B such that $AB - BA = I_n$.
    \begin{align}
        \implies tr(AB - BA) &= tr(I_n) \\
        \implies tr(AB) - tr(BA) &= n \\
        \implies 0 &= n
    \end{align}
    But it is given that $n \geq 2$. This contradicts our assumption. Hence, option 1 is FALSE. 
    
    \item   Let A be a zero matrix of order n and B be any non-diagonalizable matrix of order n. We know that a zero matrix is always diagonalizable. 
    \begin{align}
        \intertext{As, A = $0_{ n\times n }$}
        AB = 0 = BA 
    \end{align}
    We can observe here that when $AB = BA$, matrix A is diagonalizable even though matrix B is not diagonalizable. Hence, option 2 is FALSE.
    
    \item The minimal polynomial m(x), of a matrix A, is the polynomial P of minimum degree such that P(A) = 0.
    \begin{align}
        \text{Let } A &= \myvec{0 & 1 \\ 0 & 0} \text{ and } B = \myvec{0 & 0 \\ 0 & 1} \\
        \implies AB &= \myvec{0 & 0 \\ 0 & 1} \text{ and } BA = \myvec{0 & 0 \\ 0 & 0}
    \end{align}
    We can observe here that 
    \begin{align}
        m_{BA}(x) &= x \text{ while } \\
        m_{AB}(x) &= x^2
    \end{align}
    As minimal polynomial of AB and BA are not same, option 3 is also FALSE. 
    
    \item Let $\lambda$ be an eigen value of AB, it implies that there is some $x \neq 0$ such that,
    \begin{align}
        ABx = \lambda x \label{eq1}
    \end{align}
    Let y = Bx. Then $y \neq 0$. Otherwise we would get from \eqref{eq1} that $\lambda = 0$ or $x = 0$. Now we have,
    \begin{align}
        BAy &= BABx = B(ABx)  \nonumber \\
         = B(\lambda x) &= \lambda Bx = \lambda y \nonumber
    \end{align}
    It follows that $\lambda$ is an eigen value of BA. Hence, option 4 is TRUE.
\end{enumerate}

\end{document}